\documentclass[a4paper,11pt]{scrartcl}

\usepackage{graphicx}
\usepackage[utf8]{inputenc} %-- pour utiliser des accents en français
\usepackage{amsmath,amssymb,amsthm} 
\usepackage[round]{natbib}
\usepackage{url}
\usepackage{xspace}
\usepackage[left=20mm,top=20mm]{geometry}
\usepackage{algorithmic}
\usepackage{subcaption}
\usepackage{mathpazo}
\usepackage{booktabs}
\usepackage{hyperref}
\usepackage{amsmath}
% \usepackage{draftwatermark}

\newcommand{\ie}{ie}
\newcommand{\eg}{eg}
\newcommand{\reffig}[1]{Figure~\ref{#1}}
\newcommand{\refsec}[1]{Section~\ref{#1}}

\setcapindent{1em} %-- for captions of Figures

\renewcommand{\algorithmicrequire}{\textbf{Input:}}
\renewcommand{\algorithmicensure}{\textbf{Output:}}


\title{Physics Cup 2026}
\date{\today}


\begin{document}

\maketitle

%%%
%
\section{Hint }

Express the angular distance $2\alpha$ between the two foci of a hyperbola, as seen from an arbitrary point on the hyperbola, 
in terms of the angle $\beta$ of the tangent line at that point and the eccentricity $\epsilon$ of the hyperbola.

\section{Hint}
The first step towards the solution is very similar to the \href{https://eupho.ee/wp-content/uploads/2023/06/EuPhO_2023_theo_sol.pdf}{T2 solution} of the European Physics Olympiad Problem No 2 from 2023.

\section{Hint: (More Detailed Explanation to Follow Hint 01)}
There is a simple relationship between $cos\alpha$ and $cos\beta$ that can be obtained by considering four applications of 
the law of cosines for two suitably chosen triangles.

\section{Hint}
The first step is to figure out what is the brick's \href{https://en.wikipedia.org/wiki/Hodograph}{hodograph} looks like. It is
also worth consulting \href{https://eupho.ee/wp-content/uploads/2023/06/EuPhO_2023_theory.pdf}{Problem No.2 from the 2023 European Physics Olympiad}. \\

Second, you want to integrate $v_{y}$ over time $l_{y} = \int u_{y} dt = \int u_{y} \frac{u_{x}}{a_{x}}$, where the x component of the 
acceleration $a_{x}$ can be expressed in terms of  $a_{0}$ and the relevant angles. 

\section{Hint: The "Final" Boss}
Here are the steps to solve the problem.\\

First, determine the brick’s \href{https://en.wikipedia.org/wiki/Hodograph}{hodograph} and describe its shape with a formula in Cartesian coordinates (this will be needed later). You may find it helpful to 
consult the \href{https://eupho.ee/wp-content/uploads/2023/06/EuPhO_2023_theo_sol.pdf}{T2 solution} of the European Physics Olympiad Problem No 2 from 2023. \\

Second, express the force exerted by each plate in terms of the given quantities. \\

Third, express the displacement as an integral of $v_{y}$ over time $l_{y} = \int u_{y} dt = \int u_{y} \frac{u_{x}}{a_{x}}$, where the x component of the 
acceleration $a_{x}$ can be expressed in terms of  $a_{0}$ and the relevant angles. \\ 

Fourth, using the hodograph’s formula, express the integrand in terms of either $u_{x}$ or $u_{y}$. Note that integration by parts may be helpful here.\\

Fifth, solve the integral, then submit Answer to \href{mailto:physcs.cup@gmail.com}{Physics Cup Email}\\

\bibliographystyle{plainnat}
\bibliography{/Users/hugo/references/references}

\end{document}